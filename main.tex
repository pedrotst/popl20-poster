\documentclass[25pt,
a0paper,
portrait,
svgnames,
blockverticalspace=0.3cm
]{tikzposter}
%\usepackage{extsizes}
\usepackage{xcolor}
\usepackage{graphicx}
\usepackage{listings}
\usepackage{listingsutf8}
\usepackage{verbatim}
\usepackage{fancyvrb}
\usepackage{amsmath,amsfonts,amsthm,bm} % Math packages
\usepackage{CedilleVerbatim}
\usepackage{url}


\input{lstcoq.sty}

\newcommand{\inlinecoq}[1]{\lstinline[breaklines=true, language=Coq]|#1|}
\tikzposterlatexaffectionproofoff

\let\thempfootnote\thefootnote%http://tex.stackexchange.com/questions/956/footnotemark-and-footnotetext-in-minipage#959
\newcommand\printfootnote[1]{
\addtocounter{footnote}{1}%
\footnotetext{#1}}

\title{How small can we make a useful type theory?}
\author{Pedro da Costa Abreu Jr.}
\institute{Purdue University, USA \\ pdacost@purdue.edu}

% \usetheme{Desert}
\usetheme{Simple}

\begin{document}

\maketitle

% \node[anchor=east] at (TP@title.east) {\includegraphics[width=0.20\linewidth]{images/purdue-logo.png}};

% \block{The One Million Dollar Question}{
% \fontsize{25pt}{60pt}\selectfont

\begin{columns}

\column{.45}
\block{Are you also hearing Ode to Joy?}{
\textbf{Universe Hierarchy}, \textbf{datatypes} and \textbf{large elimination} are well known features 
of modern theorem provers. However, they are also known to
\textbf{complicate metatheorical properties}, such as soundness, type inference, and
normalization.

This work aims to assess how much these features
are actually necessary in practice. To achieve this we are building
\textbf{Coquedille, a compiler from a subset of Gallina to Cedille\textsuperscript{1}}, and
we will probe it by translating as much as possible of the Software Foundations Series.

% The main goal of this work is to determine how much these features are actually useful
% in practice. We aim to achieve this by building \textbf{Coquedille, a compiler
% from Coq to Cedille}.
}
\block{What do you mean Universe Hierarchy?}{
Recall Russell's paradox: 
Imagine the set of all sets that does not include itself
$$ \Delta = \{ x \ | \ x \notin x \}$$
The million dollar question is: $\Delta \in \Delta$?

If $\Delta \in \Delta$ then $\Delta \notin \Delta$,
and conversely, if $\Delta \notin \Delta$ then $\Delta \in \Delta$!.
This contradiction indicates that such $\bm{\Delta}$ \textbf{simply cannot exist}.

This means that a sound theory needs to disallow such definitions. \textbf{One fix to this problem is to only allow definitions to quantify over terms smaller than what is being defined.}
This gives rise to a \textit{hierarchy} of larger and larger types.
In Coq this is Prop, Set, Type(1), Type(2), ...

Another fix to this problem is to disallow these definitions altogether, resulting in a weaker theory.
}

\block{Cool, what about Large Eliminations?}{
\textbf{Large Eliminations are functions that computes types from terms.}
The canonical example is computing a proposition from an arbitrary term, such as:
\lstinputlisting[language=Coq]{bool_to_prop.v}
See right for an example of these in action.
}

\block{By Datatypes you mean Algebraic Datatypes, right?}{
Yep! Here are two cherrypicked examples for you:

\begin{tabular}{ll}
    \begin{minipage}[t]{.5\linewidth}
        \lstinputlisting[language=Coq]{vec.v}
    \end{minipage}
    &
    \begin{minipage}[t]{.5\linewidth}
        \lstinputlisting[language=Coq]{eq.v}
    \end{minipage}
\end{tabular}

}

 \block{So you are saying Cedille has none of this?}{
 Right. Cedille has \textbf{no large eliminations}, and \textbf{no primitive datatypes}, and like Haskell and Agda,
 \textbf{no universe universe hierarchy}\\
 
 \textit{Ask me how can we get away without datatypes.}
 }
 
 
\block{Let's talk about Coquedille}{
Coquedille is a translater from Coq to Cedille. \textbf{We implement Coquedille
in Coq} itself, this allows us to \textbf{use metacoq\textsuperscript{2}} to
access the abstract syntax tree of the gallina terms.

\textbf{The translation is mostly straightforward}.
The main challenge is that, in Coq, kinds, types and terms are syntatically the same. In Cedille we have to clearly differentiate between them. With this in mind
we access the metacoq AST and translate it to the Cedille
counterpart by using the monadic constructs provided by ext-lib\textsuperscript{3}.

We aim to measure the effectivenes of the tool by translating as many theorems as possible from the Software Foundation Series \textsuperscript{4}.


    % here we talk about useful. How far we can get with these features. What is our vehicle?

    % Well, people like Coq, so we can probe existing coq developing if they have these stuff, we have started implementing coquedille to answer this question.
}


% \block{Wait a second... Extrinsic Type Theory?}{
% Oh right, I knew I was forgeting something.
% There is one more important mismatch between Coq and Cedille: \textbf{Coq is an Intrinsic Type theory},
% but \textbf{Cedille is an Extrinsic Type Theory}.

% The key difference is that in an Intrinsic Type Theory we would write $\bm{\lambda x: T . e}$,
%  while in an Extrinsic Type Theory, it would be $\bm{\lambda x . e}$.
% I know, it seems like a minute difference but it leads to huge consequences, e.g. type uniqueness and decidability of type checking.
% In any case, this mismatch opens up the interesting question of how these two flavours of type theories will
% interact with one another soundly.
% }


\column{.03}

\column{.47}
\block{Ok... And what can you do with it?}{


\begin{tabular}[t]{p{.6\linewidth}r}
    \begin{minipage}[t]{.9\linewidth}

      Here is an example for you: let's \textbf{translate the following non-trivial proof:
      List nil and Vector vnil are not the same}.

    Since we are comparing terms of two different types we cannot use
    the regular equality, so we use the JMeq heterogeneous equality instead (notated as {\small'$\sim=$'}).

    Don't waste your time trying to understand this proof! All I want you to notice is that 
    \textbf{the highlighted part is a large elimination}.\\

    \textit{If you want to see the proof written with tactics, take a look at the project github\textsuperscript{4}}.

    \end{minipage}
    &
    \begin{minipage}[t]{0.50\linewidth}
    \lstinputlisting[language=Coq]{nil_neq_vnil.v}
    \end{minipage}
\end{tabular}

}


 \block{Hey! You just said Cedille doesn't support Large Eliminations!}{
 And it doesn't! But worry not, not all is doomed.
 We use a nice trick of translating it to
 a Cedille term $\bm{\delta}$, which basically 
 \textbf{checks if two lambda terms are \bm{$\beta\eta$}-equivalent} via the
 Bohm-out algorithm\textsuperscript{5}. If the terms being compared are not $\beta\eta$-equivalent, $\delta$
 allows us to apply the explosion principle
 %redundancy here is intentional%
 to prove anything.
 %------------------------------%
 In order to do this translation, we first prove a lemma
 relating Coq equality \lstinline[language=Coq, basicstyle=\large]{eq} to Cedille
 built-in equality \Verb[fontsize=\normalsize]+\{x ≃ y\}+\\
 
 \VerbatimInput[fontsize=\small,commandchars=+\^\&]{eqprimeq.ced}
 
Now we can translate the large elimination by
using $\delta$ on the generated proof that \lstinline[language=Coq, basicstyle=\large]{nil = cons a nil} (look at the type of H4 at the proof above).\\
\VerbatimInput[fontsize=\small,commandchars=+\^\&]{cedille-proof.ced}
}

\block{Next Steps}{
\begin{itemize}
    \item Translate induction and recursion
    \item Check how much of Software Foundations we are able
    to translate even with these shortcomings
    \item Prove the correctness of the translation
    \begin{itemize}
      \item Translation Validation
      \item Logical Relations
      \item ???
    \end{itemize}
\end{itemize}
}

\block{Wait, I still have a question}{
  I'm more than happy to hear any insights, doubts, feedbacks, or I don't know, ``hi''s.
}

% \block{Shameless Plug}{
% I'm looking for summer internships!
% }

\block{Useful links and citations}{
\textsuperscript{1} {\mbox{\url{https://github.com/cedille/cedille}}}.

\textsuperscript{2} {\mbox{\url{https://github.com/MetaCoq/metacoq}}}.

\textsuperscript{3} {\mbox{\url{https://softwarefoundations.cis.upenn.edu/}}}.

\textsuperscript{3} {\mbox{\url{https://github.com/coq-community/coq-ext-lib}}}.

\textsuperscript{4} {\mbox{\url{https://github.com/pedrotst/coquedille}}}.

\textsuperscript{5} H. Barendregt, The Lambda Calculus; Its Syntax and Semantics.


}

\end{columns}

\end{document}

%%% Local Variables:
%%% mode: latex
%%% TeX-master: t
%%% End:

\documentclass[25pt, a0paper, portrait, svgnames]{tikzposter}
%\usepackage{extsizes}
\usepackage{xcolor}
\usepackage{graphicx}
\usepackage{listings}
\usepackage{listingsutf8}
\usepackage{lstlinebgrd}
\usepackage{verbatim}
\usepackage{fancyvrb}
\usepackage{amsmath,amsfonts,amsthm,bm} % Math packages
\usepackage{CedilleVerbatim}
\usepackage{url}


\input{lstcoq.sty}

\newcommand{\inlinecoq}[1]{\lstinline[breaklines=true, language=Coq]|#1|}
\tikzposterlatexaffectionproofoff

\let\thempfootnote\thefootnote%http://tex.stackexchange.com/questions/956/footnotemark-and-footnotetext-in-minipage#959
\newcommand\printfootnote[1]{
\addtocounter{footnote}{1}%
\footnotetext{#1}}

\title{How small can we make a useful type theory?}
\author{Pedro da Costa Abreu Jr.}
\institute{Purdue University, USA \\ pdacost@purdue.edu}

% \usetheme{Desert}
\usetheme{Simple}

\begin{document}

\maketitle

% \node[anchor=east] at (TP@title.east) {\includegraphics[width=0.20\linewidth]{images/purdue-logo.png}};

% \block{The One Million Dollar Question}{
% \fontsize{25pt}{60pt}\selectfont

\begin{columns}

\column{.45}
\block{There are too many posters here, why should I read yours?}{
\textbf{Universe Hierarchy}, \textbf{datatypes} and \textbf{large elimination} are well known features 
of modern theorem provers. However, they are also known to
\textbf{complicate metatheorical properties}, such as soundness, type inference, and
normalization.

This work aims to assess how much these features
are actually necessary in practice. To achieve this we build
\textbf{Coquedille, a compiler from a subset of Gallina to Cedille}, and
probe it by translating as much as possible of the Software Foundations Series.

% The main goal of this work is to determine how much these features are actually useful
% in practice. We aim to achieve this by building \textbf{Coquedille, a compiler
% from Coq to Cedille}.
}
\block{What do you mean Universe Hierarchy?}{
Recall Russell's paradox: 
Imagine the set of all sets that does not include itself
$$ \Delta = \{ x \ | \ x \notin x \}$$
The million dollar question is: $\Delta \in \Delta$?

If $\Delta \in \Delta$ then $\Delta \notin \Delta$,
conversely, if $\Delta \notin \Delta$ then $\Delta \in \Delta$!.
This contradiction indicates that such $\bm{\Delta}$ \textbf{simply cannot exist}.

This means that a sound theory needs to disallow such definitions. \textbf{One fix to this problem is to only allow definitions to quantify over terms smaller than what is being defined.}
This gives rise to a \textit{hierarchy} of larger and larger types.
In Coq this is Prop, Set, Type(1), Type(2), ...

Another fix to this problem is to disallow these definitions altogether, resulting in a weaker theory.
}

\block{Cool, what about Large Eliminations?}{
\textbf{Large Eliminations are function that computes types from terms.}
The canonical example is computing a proposition from an arbitrary term, such as:
\lstinputlisting[language=Coq]{bool_to_prop.v}
See right for an example of these in action.
}

\block{By Datatypes you mean Algebraic Datatypes, right?}{
Yep! Here are two cherrypicked examples for you:

\begin{tabular}{ll}
    \begin{minipage}[t]{.5\linewidth}
        \lstinputlisting[language=Coq]{vec.v}
    \end{minipage}
    &
    \begin{minipage}[t]{.5\linewidth}
        \lstinputlisting[language=Coq]{eq.v}
    \end{minipage}
\end{tabular}

}

 \block{So you are saying this Cedille has none of this?}{
 Exactly! Cedille has \textbf{no large eliminations}, and \textbf{no primitive datatypes}, and like Haskell and Agda,
 \textbf{no large eliminations}, all of this implemented in an extrinsic type theory.\\
 
 \textit{Ask me how can we get away without datatypes.}
 }
 
 
% \block{Let's talk about Coquedille}{
% Coquedille is a translater from Coq to Cedille using metacoq
% 
% here we talk about useful. How far we can get with these features. What is our vehicle?
% 
% Well, people like Coq, so we can probe existing coq developing if they have these stuff, we have started implementing coquedille to answer this question.
% 
% }


\block{Wait a second... Extrinsic Type Theory?}{
Oh right, I knew I was forgeting something.
There is one more important mismatch between Coq and Cedille: \textbf{Coq is an Intrinsic Type theory},
but \textbf{Cedille is an Extrinsic Type Theory}.

The key difference is that in an Intrinsic Type Theory we would write $\bm{\lambda x: T . e}$,
 while in an Extrinsic Type Theory, it would be $\bm{\lambda x . e}$.
I know, it seems like a minute difference but it leads to huge consequences, e.g. type uniqueness and decidability of type checking.
In any case, this mismatch opens up the interesting question of how these two flavours of type theories will
interact with one another soundly.
}


\column{.03}

\column{.47}
\block{All sounds nice and good, but do you have any results yet?}{


\begin{tabular}[t]{p{.6\linewidth}r}
    \begin{minipage}[t]{.9\linewidth}
    \textbf{We implement Coquedille in Coq} itself. And we get access to the syntax of the terms by \textbf{using Metacoq}.
    
    
Let me show you things working by \textbf{translating a non-trivial proof: List nil and Vector vnil are not the same}.

Since we are comparing terms of two different types we cannot use 
the regular equality, so we use the JMeq heterogeneous equality instead (notated as {\small'$\sim=$'}).

    Don't waste your time trying to understand this proof! All I want you to notice is that 
    \textbf{the highlighted part is a large elimination}.\\
    
    \textit{If you want to see the proof written with tactics, take a look at the project github at\footnotemark{}}
    
    \footnotemark{}\footnotemark{}
    \footnotemark{}\footnotemark{}
    \footnotemark{}\footnotemark{}
    
    \end{minipage}
    &
    \begin{minipage}[t]{0.50\linewidth}
    \lstinputlisting[language=Coq]{nil_neq_vnil.v}
    \end{minipage}
\end{tabular}

}


 \block{Hey! You just said Cedille doesn't support Large Eliminations!}{
 Preciselly! How can we translate it then?
 We use a nice trick of translating it to
 a Cedille term $\bm{\delta}$, which basically 
 \textbf{checks if two lambda terms are \bm{$\beta\eta$}-equivalent} via the
 Bohm-out algorithm. If the terms being compared are not $\beta\eta$-equivalent, $\delta$
 allows us to apply the explosion principle to prove anything.
 In order to do this translation, we first prove a lemma
 relating Coq equality \lstinline[language=Coq, basicstyle=\large]{eq} to Cedille
 built-in equality \Verb[fontsize=\normalsize]+\{x ≃ y\}+\\
 
 \VerbatimInput[fontsize=\small,commandchars=+\^\&]{eqprimeq.ced}
 
Now we can translate the large elimination by
using $\delta$ on the generated proof that \lstinline[language=Coq, basicstyle=\large]{nil = cons a nil} (look at the type of H4 at the proof above).\\
\VerbatimInput[fontsize=\small,commandchars=+\^\&]{cedille-proof.ced}
}
\block{Next Steps}{
\begin{itemize}
    \item Translate induction and recursion
    \item Check how much of Software Foundations we are able
    to translate even with these shortcomings
    \item Prove the correctness of the translation
\end{itemize}
}

\block{Questions?}{
Insights? Inconsistencies? Doubts? Feedback?
}

% \block{Shameless Plug}{
% I'm looking for summer internships!
% }

\block{Citations}{
\setcounter{footnote}{0}%
\printfootnote{ipsum} % ipsum is the text of the footnote
    
\printfootnote{\mbox{\url{https://github.com/pedrotst/coquedille}}}.
}

\end{columns}

\end{document}